% LaTeX Template for short student reports.
% Citations should be in bibtex format and go in references.bib
\documentclass[11pt]{article}
\usepackage[top=3cm, bottom=3cm, left = 2cm, right = 2cm]{geometry}  
\usepackage[utf8]{inputenc}
\usepackage{textcomp}
\usepackage{graphicx} 
\usepackage{amsmath,amssymb}  
\usepackage{bm}  
\usepackage[pdftex,bookmarks,colorlinks,breaklinks]{hyperref}  
\hypersetup{linkcolor=black,citecolor=black,filecolor=black,urlcolor=black} % black links, for printed output
\usepackage{memhfixc} 
\usepackage{pdfsync}  
\usepackage{fancyhdr}
\pagestyle{fancy}

\title{Responsible Data Science Course Project}
\author{Claire Saint-Donat, Xiangyue Wang}
\date{Spring 2022}

\begin{document}
\maketitle
\tableofcontents

\section{Introduction}

In this project, we attempt to build an interpretability tool for an Automated Decision System (ADS). 

Automated Decision Systems (ADS) are in widespread use in government and industry, and a number of efforts are currently underway to regulate them.  New York City recently passed a law (Local Law 49 of 2018) that compels the development of procedures and recommendations that City agencies should follow when explaining the operation of an ADS to the public, and demonstrating that an ADS does not discriminate against individuals based on membership in protected groups.  

In this project, we attempt to help NYC and other municipalities by designing a``nutritional label" (similar to a label used to evaluate food products) for an algorithmic system of our choice. 
\pagebreak

\section{Background}
The Automated Decision System (ADS) that we chose to examine is from the Human Resources and Hiring space.  In particular, we wish to examine a top entry from an HR Analytics Kaggle competition published here.  The source of these data come from an active company in the Big Data and Data Science space that is looking to hire data scientists amongst a group of candidates who have taken training courses hosted by the company.  Many candidates sign up for training and the firm would like to know which trainees would be interested in working for the company or which are currently looking for new employment.  Targeting these candidates reduces costs and time associated with hiring data scientists.  In addition, a model of this kind would also help improve the quality of these trainings and the planning of courses that might help categorize candidates.  The desired outcome of this model is to accurately predict the probability that a given trainee is looking for a new job as well as the probability that a candidate would be interesting in working for the company.  The ultimate goal of this ADS is to determine which trainees should be invited in interview for data science positions at the company.

Possible trade-offs - candidates looking for a new job might not necessarily be interested in working for the company.  Candidates looking for a new job and interested in the company might not necessarily be the best candidates for DS roles at the company.

\begin{itemize}
\item What is the purpose of this ADS and it's stated goals
\item If the ADS has multiple goals, explain any trade-offs that these goals may introduce.
\end{itemize}


\pagebreak

\section{Input and Output}

These data, made available by the company, are designed to understand the factors that lead a data scientist to search for a new job.  Important information pertaining to demographics, education and prior experience is made available (on an anonymized basis) from registered and enrolled candidates.  There is little information available about the geography of the candidate population in question or even the time during which this data was collected however we can gather than \ldots

\begin{itemize}
	\item Describe the data used by this ADS.  How was this data collected or selected?
	\item For each input feature, describe its datatype, give information on missing values and on the value distribution. Show pairwise correlations between features if appropriate.  Run any other reasonable profiling of the input that you find interesting or appropriate
	\item What is the output of the system (e.g. is it a class label, a score, a probability or some other type of output) and how do we interpret it?
\end{itemize}

\pagebreak

\section{Implementation and Validation}
Present your understanding of the code that implements this ADS.  This code was implemented by others in this part of the assignment.  Your goal here is to demonstrate that you understand the implementation at a high level.
\begin{itemize}
\item Describe data cleaning and any other pre-processing
\item Give high-level information about the implementation of the system
\item How was the ADS validated?  How do we know that it meets its stated goal(s)?
\end{itemize}

\pagebreak

\section{Outcomes}

\begin{itemize}
\item Analyze the effectiveness (accuracy) of the ADS by comparing it's performance across subpopulations
\item Select one or more fairness or diversity measures, justify your choice of these measures for the ADS in question and quantify the fairness or diversity of this ADS.
\item Develop additional methods for analyzing ADS performance: think about stability, robustness, performance on difficult or otherwise important examples (in the style of LIME) or any other property that you believe is important to check for this ADS
\end{itemize}

\pagebreak

\section{Conclusion \& Summary}
\begin{itemize}
\item Do you believe that the data was appropriate for this ADS?  
\item Do you believe the implementation is robust, accurate and fair?  Discuss any choice of accuracy and fairness measures and explain which stakeholders may find these measures appropriate.
\item Would you be comfortable deploying this ADS in the public sector, or in industry?  Why so or why not?
\item What improvements do you recommend to the data collection, processing or analysis methodology?
\end{itemize}
\bibliographystyle{abbrv}
% \bibliography{references}  % need to put bibtex references in references.bib 
\end{document}